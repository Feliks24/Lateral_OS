\documentclass{article}

\input{structure.tex}
\newcommand{\sheetNR}{02}

\usepackage{graphicx}

\begin{document}
\hrule\medskip\rule{0ex}{0ex}\\[-1ex]\noindent\large
\textbf{Betriebssysteme: Sheet \sheetNR \space}\\[0.5ex]
\normalsize{Feliks \& Bennet}

\medskip\hrule\noindent\vskip 0.5cm

%----------start of document

\section*{Aufgabe \sheetNR-1(Threads)}

% * Aufgabenstellung --------------------------------------------------------------------------------------------------------------------------------------

\noindent\textit{1.Wann is es sinnvoll, nebenläufige Programmteile mit Hilfe von Threads
anstatt Prozessen zu implementieren?}

Dies und das \cite{gentry2008}

\noindent\textit{2.Welche Vorteile bieten User-Level-Threads gegenüber den
Kernel-Level-Threads? Gibt es auch Nachteile?}

Lösung 2

\noindent\textit{3.Welche Vorteile und Nachteile gibt es, wenn man
Thread-Kntrolblücke(TCB) als Skalare, in Arrays, Listen, Bäumen oder
invertierten Tabellen speichert?}

Lösung 3


\noindent\textit{4.In welchem Adressraum wird ein TCB gespeichert?}

Lösung 4

% * -------------------------------------------------------------------------------------------------------------------------------------------------------
\newpage
\printbibliography

\end{document}
