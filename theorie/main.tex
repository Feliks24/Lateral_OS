\documentclass{article}

\input{structure.tex}
\newcommand{\sheetNR}{02}

\usepackage{graphicx}

\begin{document}
\hrule\medskip\rule{0ex}{0ex}\\[-1ex]\noindent\large
\textbf{Betriebssysteme: Sheet \sheetNR \space}\\[0.5ex]
\normalsize{Feliks \& Bennet}

\medskip\hrule\noindent\vskip 0.5cm


%----------start of document

\section*{Aufgabe \sheetNR-1(Threads)}

% * Aufgabenstellung --------------------------------------------------------------------------------------------------------------------------------------

\subsubsection*{1.Wann is es sinnvoll, nebenläufige Programmteile mit Hilfe von Threads
anstatt Prozessen zu implementieren?}

% Dies und das \cite{gentry2008}
% ? https://www.ibm.com/docs/de/aix/7.3.0?topic=programming-understanding-threads-processes
% ? https://www.youtube.com/watch?v=TGgUEamLvGs

\noindent
Ein Programm (Prozess) besitzt mindestens einen Thread. 
Die Nutzung von weiteren Threads ist dann sinnvoll, wenn eine schnellere Prozessausführung innerhalb des selben Prozesses gewünscht ist, 
welche den gleichen Adressraum teilen. Diese ist bei einem Multikernsystem durch parallelisierung von mehrern Threads möglich. 
Durch die leichtgewichtigkeit eines Threads kann es schneller zur Erstellung und gewünschten Kontextwechsels kommen. Durch das Teilen des Adressraums 
können Daten untereinander ausgetauscht werden.

\subsubsection*{2. Welche Vorteile bieten User-Level-Threads gegenüber den Kernel-Level-Threads? Gibt es auch Nachteile?}

% ? https://www.youtube.com/watch?v=sCDRghC4_AI&t=232s
% ? https://www.tutorialspoint.com/user-level-threads-and-kernel-level-threads
% ? https://www.ibm.com/docs/kk/aix/7.2.0?topic=processes-kernel-threads-user-threads

\noindent
Die Funktion der beiden Thread-Arten unterscheidet sich, daher gehen die eigentlichen Vor- und Nachteile aus der Verwendung hervor.
Voreile eines User-Level-Threads für die Enwicklung ist die Lenkbarkeuit und der Zugriff über eine einheiltlcihe Thread-Libary. Die erstellten
Threads sind schneller, einfacher zu erstellen zu steuern und zu mangen. Sie funktionieren Plattformunabhängig und benötigen keine Kernel mode priviliegien 
um einen Kontextwechsel vorzunehmen. 
Das Kernel weiß nichts von den verschiedenen Threds aus User-Level und behandelt die Prozesse wie eine eine single-Thread-Application, das kann zu Nachteilen führen. Das 
Multithreading kann nicht zum Vorteil der Anwendung genutzt werden. Falls ein User-Level-Thread eine Blockieroperation, wird der gesammte Prozess blockiert. Dies kann bei einem 
Kernel-Level-Thread nicht passieren Multithreading könenn über das Kernel geschedulart werden und bei der Blockierung eines Threads, kann ein anderer Thread innerhalb des Prozesses mit der 
Operastion vortfahren.

\subsubsection*{3.Welche Vorteile und Nachteile gibt es, wenn man Thread-Kontrollblock(TCB) als Skalare, in Arrays, Listen, Bäumen oder invertierten Tabellen speichert?}

\medskip
\noindent
\begin{table}[h!]
\centering
\begin{tabular}{l p{5cm} p{6cm}}
\hline
\textbf{Struktur} & \textbf{Vorteil} & \textbf{Nachteil} \\
\hline
Skalar & Einfache Implementierung & Unterstützt keine Mehrfach-Threads \\
Array & Schneller Indexzugriff, gut für feste Prioritäten & Unflexibel, schlecht bei dynamischer Anzahl \\
Liste & Einfaches Einfügen/Löschen & Langsame Suche, langsam bei prioritätsbasiertem Scheduling \\
Baum & Sehr gut für Prioritäten, schnelle Operationen (O(log n)) & Komplexe Implementierung, zusätzlicher Overhead \\
Invertierte Tabelle & Schneller Lookup über ID → TCB & Benötigt Hashing oder direkte Indexierung, Kollisionsgefahr \\
\hline
\end{tabular}
\caption{Vergleich von Datenstrukturen zur Speicherung von TCBs}
\end{table}



\subsubsection*{4.In welchem Adressraum wird ein TCB gespeichert?}

\noindent
Der TCB eines Kernel-Threads wird im Kernel-Adressraum gespeichert, typischerweise im Kernelspeicher des Betriebssystems. Sie können auch durch Kerneloperationen gesteuert und gemanged werden.

% * -------------------------------------------------------------------------------------------------------------------------------------------------------
\newpage
%\printbibliography

\end{document}
